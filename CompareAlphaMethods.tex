\documentclass[fleqn]{article}
\usepackage[utf8]{inputenc}
\usepackage[a4paper, total={6in, 8in}]{geometry}
\usepackage{natbib}
\usepackage{graphicx}
\usepackage{amsmath}
\setlength{\parindent}{4em}
\renewcommand{\baselinestretch}{2}

\title{Comparing empirical methods measuring intra- ($\alpha_{ii}$) and inter-specific interaction coefficient}
\author{Feng-Hsun Chang}
\date{April 2018}

\begin{document}

\maketitle

% \section*{Introduction}

% Contemporary coexistence theory is established on Chesson's key insight toward the mutual invasibility criterion of the classic Lokta-Volterra competition model \cite{Chesson1990, Chesson2000, Chesson2013, Chesson2008, Adler2007}. 


% \begin{equation}\label{eq:1}
%  \frac{dN_1}{dt} = r_1N_1(1 - \alpha_{11}N_1 - \alpha_{12}N_2)
% \end{equation}

% \begin{equation}\label{eq:2}
%  \frac{dN_2}{dt} = r_2N_2(1 - \alpha_{21}N_1 - \alpha_{22}N_2)
% \end{equation}


% In the classic Lokta-Volterra competition model, species are guaranteed to coexist when intra-specific interaction is greater than inter-specific interaction, i.e. $\alpha_{11} > \alpha_{12}$ and $\alpha_{22} > \alpha_{21}$. Chesson showed that this mutual invasibility criterion can be defined in terms that quantify the degree of niche overlap, $\rho$ between species and their difference between their average fitness, $\frac{f_2}{f_1}$ ($f_i$ is the same as the $k_i$ in Chesson's original paper). Chesson defines niche overlap ($\rho$) as $\sqrt{\frac{\alpha_{12} \alpha_{21}}{\alpha_{11} \alpha_{22}}}.$ The relative fitness difference (RDF; $\frac{f_2}{f_1}$ ) is defined as $\frac{\alpha_{12}}{\alpha_{22}}\frac{1}{\rho}$; therefore $ \frac{f_2}{f_1} = \sqrt{\frac{\alpha_{12}\alpha_{11}}{\alpha_{21}\alpha_{22}}}.$

% From the above definition and deduction, the ratio of inter- specific to intra- specific competition coefficients is equal to the product of a niche overlap term ($\rho$) and a fitness ratio term ($\frac{f_2}{f_1}$)

% \begin{equation}\label{eq:3}
%  \frac{\alpha_{12}}{\alpha_{22}} = \frac{f_2}{f_1}\rho
% \end{equation}

% \noindent From the mutual invasibility criterion, we know that the right hand side of Eq. 2 must be $<1$ for species 1 to be able to invade a population of species 2 at equilibrium. This is the same as saying $f_2/f_1 < 1/\rho$. By the same logic, for species 2 to be able to invade a population of species 1, $f_1/f_2 < 1/\rho$. Therefore, satisfaction of the mutual invasibility criterion, i.e., stable coexistence, requires 

% \begin{equation}\label{eq:4}
%  \rho < \frac{f_2}{f_1} < \frac{1}{\rho}.
% \end{equation}

% Since both niche overlap term ($\rho$) and a fitness ratio term ($\frac{f_2}{f_1}$) depend on inter- and intra-specific interaction coefficient ($\alpha_{ii}$ and $\alpha_{ij}$), estimation of $\alpha$s is critical in assessing species coexistence. Conceptually, $\alpha_{ij}$ captures the \textit{per capita} impact of species $j$ on species $i$. Four empirical methods so far are proposed to estimate $\alpha_{ij}$. These four methods include (1) Letten et al.'s method \cite{Letten2017}, (2) sensitivity method \cite{Carroll2011, Narwani2013}, (3) negative frequency dependency (NFD) method \cite{Levine2009, Hillerislambers2012}, and (4) calculating $\alpha$ by fitting growth curves to population dynamics. In this article I discuss if these four methods are actually estimating $\alpha$s in the following generic Lokta-Volterra competition model. 

\section*{Introduction}

We focus on the fluctuation independent models to discuss species coexistence. 

\newpage
\section{The classic Lotka-Volterra model}

In the classic Lotka-Volterra model, the \textit{pre capita} growth rate of species $i$ can be described by the following equation. 

\begin{equation}\label{eq:1}
 \frac{1}{N_i}\frac{dN_i}{dt} = r_i(1 - \alpha_{ii}N_i - \sum_{j \neq i}^{} \alpha_{ij}N_j)
\end{equation}

\noindent In the model, $N_{i}$ is the density of species $i$, $r_i$ is the intrinsic growth rate. The $\alpha_{ii}$ is the intra-specific competition coefficient, which describes the \textit{per capita} effect of species $i$ on the \textit{per capita} growth rate of species $i$. The $\alpha_{ij}$ is the inter-specific competition coefficient, which describes the \textit{per capita} effect of species $j$ on the \textit{per capita} growth rate of species $i$. For the two species to stably coexist, the intra-specific competition coefficient must be greater than the inter-specific competition coefficient, i.e. $\alpha_{ii} > \alpha_{ij}$. 

To use this model to predict species coexistence, one must have a clean enough time series of population density to estimate the competition coefficients. The effort to obtain such time series is not trivial. In addition, fitting parameters is also not easy. 

\newpage
\section{Sensitivity method}

Following the same logic that intra-specific competition coefficient must be greater than the inter-specific one to guarantee stable coexistence, the sensitivity method is proposed \cite{Carroll2011}. The idea of sensitivity method is to estimate the reduction of species $i$'s \textit{per capita} growth rate due to the existence of the other species. This reduction is the impact of the other species on species $i$'s \textit{per capita} growth rate, which is the meaning of inter-specific competition coefficient ($\alpha_{ij}$). Specifically, according to Carroll \textit{et al.} 2011, sensitivity is calculated by the following formula. 

\begin{equation}\label{eq:2}
S_{ij} \equiv \frac{\mu_{i}-\mu_{ij}}{\mu_{i}}
\end{equation}

\noindent In equation \ref{eq:2}, $\mu_i$ is the \textit{per capita} growth rate of species $i$ when growing alone and $\mu_{ij}$ is the \textit{per capita} growth rate of species $i$ when the other species $j$ is at its carrying capacity. Since $\mu_{ij}$ is measured when the other species $j$ is at the carrying capacity, the reduction of species $i$'s \textit{per capita} growth rate, i.e. the nominator, is actually caused by the entire population of the other species $j$. Accordingly, the sensitivity measures the "population" impact of species $j$, but not the \textit{per capita} impact of species $j$. 

To show that sensitivity is actually the population level, not the \textit{per capita}, impact, we derive the sensitivity from a classic Lotka-Volterra competition model (equation \ref{eq:1}). The $\mu_{i}$ is then $r_i$ and $\mu_{ij}$ is $r_i(1-\alpha_{ij}N^*_j)$. Consequently, 

\begin{equation}\label{eq:3}
S_{ij} \equiv \frac{\mu_{i}-\mu_{ij}}{\mu_{i}} = \frac{r_i-r_i(1-\alpha_{ij}N^*_j)}{r_i} = \alpha_{ij}N^*_j
\end{equation}

\noindent From equation \ref{eq:3}, we see clearly that sensitivity ($S_{ij}$) measures the overall population effect of species ${j}$ on focal species $i$ but not the "\textit{per capita}" effect of species $j$. Small tweak should be implemented when using the sensitivity method to estimate $\alpha$s. 

However, sensitivity ($S_{ij}$) is not being used as $\alpha$s directly. When Carroll \textit{et. al} used the sensitivity ($S_{ij}$) to calculated niche difference (ND), relative fitness difference (RDF) and eventually predict species coexistence, they defined the ND as the geometric mean of $S_{ij}$ and RDF as the geometric standard deviation of $S_{ij}$. This makes intuitive sense because when species overlap their niche more, the larger the sensitivity ($S_{ij}$) should be. In addition, since geometric mean is more sensitive to changes near near 0 than changes closer to 1, using geometric mean makes the ND more sensitive to species that are less responsive to competition (i.e. species has lower $S_{ij}$) so gives more weight to these species. This definition does not have direct linkage to the Lotka-Volterra model or any mathematical deduction. Surprisingly, defining ND and RFD as the geometric mean and standard deviation of sensitivity ($S_{ij}$) accurately predict the species coexistence.  

\newpage
\section{Negative frequency dependency (NFD) method}

The negative frequency dependency (NFD) method \cite{Adler2007} is also derived from the same logic that intra-specific competition coefficient must be greater than the inter-specific competition coefficient for stable coexistence. The rational of NFD is to measure how the \textit{per capita} growth rate of a focal species $i$ would reduce with the increase of the frequency (\%) of the other species ($j$) in a community. This reduction is thus the impact of the other competing species $j$ on the \textit{per capita} growth rate of species $i$. However, when calculating the NFD, increase of the other species $j$' frequency is being represented by the decrease of species $i$'s frequency \cite{Adler2007, Levine2009, Hillerislambers2012, Godoy2014}. Lower frequency of the focal species ($i$) means the higher frequency and thus stronger impact of the competing species $j$. Accordingly, the dependency of focal species $i$'s \textit{per capita} growth rate on its frequency is actually the "per \%" impact of competing species $j$ in the community on the \textit{per capita} growth rate of a focal species $i$. The NFD is therefore not a proper measure of \textit{per capita} impact of competing species on the \textit{per capita} growth rate of focal species $i$. 

Again, to show that NFD is not a proper measurement of inter-specific competition coefficient, we attempt to derive the NFD from the Lotka-Volterra model again. We found that, the negative frequency dependency (NFD) cannot be readily derived from the Lotka-Volterra model without making further assumptions. In the Lotka-Volterra model, there is no term describing the frequency of species. The only way to make the \textit{per capita} growth rate a function of the frequency of species $i$ is to assume a one-to-one conversion between focal species $i$ and the competing species $j$ and a constant community density. By doing so, the Lokta-Volterra competition model can be rewritten as followed. 

\begin{equation}\label{eq:4}
 \frac{1}{N_i}\frac{dN_i}{dt} = r_i(1 - \alpha_{ii}N_i - \sum_{j \neq i}^{} \alpha_{ij}(1-N_j))
\end{equation}

In equation \ref{eq:4}, $N_i$ becomes the frequency, not the density, of species $i$. To calculate the negative frequency dependency (NFD), we take derivative of equation \ref{eq:4} in terms of $N_i$. 

\begin{equation}\label{eq:5}
 NFD \equiv \frac{\partial \frac{dN_i}{dt}}{\partial N_i} = r_i(\alpha_{ij} - \alpha_{ii})
\end{equation}

\noindent From equation \ref{eq:5}, the change of \textit{per capita} growth rate of focal species$i$ with respective to species $i$'s frequency should be $r_i(\alpha_{ij} - \alpha_{ii})$ but not $\alpha_{ij}$. Again, this is the "per \%", not the "\textit{per capita}" impact of competing species $j$ on the \textit{per capita} growth rate of focal species $i$. Accordingly, NFD is not a proper measurement of $\alpha_{ij}$. What is even worse is that this derivation is founded on a strong assumption that the one individual increase of competing species $j$ leads to one individual decrease of the focal species $i$ and that the community density is constant. 

% I'm here~~~~~~~~~~~~~~~~~~~~~

\newpage
\section{Letten \textit{et al.} 2017's method}

Letten \textit{et al.} 2017 reorganize the mechanistic consumer-resource model (\cite{Tilman1977}; equation \ref{eq:5} to \ref{eq:8}) to a phenological Lokta-Volterra form so that one can decipher the parameters impacting species' \textit{per capita} growth rate. 

\begin{equation}\label{eq:6}
 \frac{dN_1}{dt} = r_1\frac{R_2}{R_2+K_{12}} - DN_1
\end{equation}

\begin{equation}\label{eq:7}
 \frac{dN_2}{dt} = r_2\frac{R_1}{R_1+K_{21}} - DN_2
\end{equation}

\begin{equation}\label{eq:8}
 \frac{dR_1}{dt} = D(S_1 - R_1) - r_1\frac{R_1}{(R_1 + K_{11})y_11}N_1 - r_2\frac{R_1}{(R_1 + K_{21})y_21}N_2
\end{equation}

\begin{equation}\label{eq:9}
 \frac{dR_2}{dt} = D(S_2 - R_2) - r_1\frac{R_2}{(R_2 + K_{12})y_12}N_1 - r_2\frac{R_2}{(R_2 + K_{22})y_22}N_2
\end{equation}

According to Letten \textit{et al.} the inter- and intra-specific interaction coefficients can be expressed as following, 

\begin{equation}\label{eq:10}
 \alpha_{11} = \frac{C_{12}}{y_{21} D (S_2 - R^*_2)} 
\end{equation}

\begin{equation}\label{eq:11}
 \alpha_{12} = \frac{C_{22}}{y_{22} D (S_2 - R^*_2)} 
\end{equation}

\begin{equation}\label{eq:12}
 \alpha_{22} = \frac{C_{21}}{y_{12} D (S_1 - R^*_1)} 
\end{equation}

\begin{equation}\label{eq:13}
 \alpha_{21} = \frac{C_{11}}{y_{11} D (S_1 - R^*_1)} 
\end{equation}

\noindent In the above equations, $C_{ij}$ is the consumption of species $i$ on resource $j$, $D$ is the dilution rate, $y_{ij}$ is the yield of species $i$ per resource $j$, $S_{i}$ is the supply rate of resource $i$, and $R^*_i$ is the resource density at the equilibrium. 

Note that the above consumption term ($C_{ij}$) is a function of resource density in the generic consumer-resource model, e.g. $C_{12} = \frac{r_1R_2}{R_2 + K_{12}}$. However, if the consumption term is resource density dependent, this linkage would not hold. The linkage between consumer-resource model and Lotka-Volterra model only exist when all the consumption of species $i$ on resource $j$ are resource density independent. For example, Letten \textit{et al.}'s predicted coexistence region will only match simulation results with empirical parameter values when letting $C_{ij}$ equals to $\frac{D}{y_{ij}}$. That is to say, Letten \textit{et al.}'s method only works in a special case of consumer-resource model.  
 





\newpage
\bibliographystyle{plain}
\bibliography{CompareAlphaMethods_Ref}
\end{document}
